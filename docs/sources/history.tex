
Краткая история и биография платформы.
\par
Будучи студентом в 2010-2011 году я был в составе команды keva.
После нескольких игр я имел представление о том что такое игры ctf.
Начинал с ресурса http://hax.tor.hu/ ресурс мне понравился.
В итоге я начал думать над идеей создать платформу для игр но такую что бы можно было
ее использовать как архив заданий для тренировки новичков да и так что бы можно было выбирать задания по силам.
Также я обсуждал это идею с командой. Прошло время, после учебы решил продолжать заниматься ctf но уже в организации и подготовки команды к играм.
И все таки вернулся к своей идеи (FreeHackQuest) и написал первую версию (2012), что бы провести в университете игру.
При поддрежки Алексея Гуляева (Второго) и Витали Шишкина мы развесили объявления и набрали новых людей в команду.
В следующем году (2013) платформа была полностью переписана и проведен FHQ2013 опять же для рекрутинга, но вэтот раз играли не только с нашего университета но и с других городов.
После этого мы оставли висеть платформу в режими онлайн (идея была Алексея Гуляева), что позволило привлечь больше людей на нашу сторону.
Эта же платформа (с рядом доработок) была использована при проведении SibirCTF2014. Там были написаны модули для проведения attack-defence игр.
И вот заканчивается 2014 год дизайн был полностью сменен в очередной раз, доработки и прочее. Надеюсь что будет проведен в последний раз FHQ 2014. 
О планах потом но платформа будет жить, только уже в новых формах.


